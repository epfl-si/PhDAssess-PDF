% !TEX encoding = UTF-8
% !TEX program = Lualatex

%%%%%%%%%%%%%%%%%%%%%%%%%%%%%%%%%%%%%%%%%
%
% Doc
%
%%%%%%%%%%%%%%%%%%%%%%%%%%%%%%%%%%%%%%%%%

%----------------------------------------------------------------------------------------
%	PACKAGES AND OTHER DOCUMENT CONFIGURATIONS
%----------------------------------------------------------------------------------------

\documentclass{article}


\usepackage{luacode}

\begin{filecontents*}{test.json}
{
    "recipe": {
    "title":"First recipe",
    "source":"My first cookbook",
    "carbs":"1 oz",
    "fat":"1 oz",
    "protein":"1 oz",
    "cal":"100 kcal",
    "ingredients": [
        {"item":"Eggs"},
        {"item":"Oil"},
        {"item":"Nuts"}
    ],
    "cooking": [
        {"step":"Mix eggs and oil"},
        {"step":"Add nuts"}
    ]
}
}
\end{filecontents*}


\begin{document}

%----------------------------------------------------------------------------------------
%	SUMMARY SECTION
%----------------------------------------------------------------------------------------
    \title{Introduction to \LaTeX{}}
    \author{Author's Name}

    \maketitle

    \begin{abstract}
        The abstract text goes here.
    \end{abstract}

    \section{Introduction}
    Here is the text of your introduction.

    \begin{equation}
        \label{simple_equation}
        \alpha = \sqrt{ \beta }
    \end{equation}

    \subsection{Subsection Heading Here}
    Write your subsection text here.

    \begin{figure}
        \centering
        \caption{Simulation Results}
        \label{simulationfigure}
    \end{figure}

    \section{Conclusion}
    Write your conclusion here.

\end{document}
